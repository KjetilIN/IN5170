\section{Async Message Passing}

In the following we assume that messages are never lost, but can arrive in a different order than
they are sent.

\subsection{Problem 12: Actor with Erlang Style Code}

Write two actors using the code skeleton below that implements the
following behavior. For each state of the actor, use a different loop: 

\begin{quote}
    You are designing a web application with two components: GUI and Backend. The
    backend stores a single value that can be set and retrieved through the GUI. The GUI
    is either WAITING or RESPONSIVE. If it is RESPONSIVE, it accepts messages
    of the form (GET, user) and (SET, n). In the first case, it changes its state to
    WAITING, stores the user and sends a message to the backend to retrieve the stored
    value. In the latter case, it sends a message to the backend to store the value n. If
    it is WAITING, it only accepts messages of the form (VALUE,n), send the value n
    to the stored user. The backend accepts (SET, n) messages to update its stored value and answers
    (GET) messages by sending the stored value to the GUI.
\end{quote}

\textbf{Solution}

\begin{lstlisting}[language=erlang]
handleGUIRequest(state, user) -> 
    receive
        {from, GET, user} -> {

            handleGUIRequest(WAITING, user)

        },
        {from, SET, n} -> {

        }


handleBackendRequest(value) -> 
    receive 
        {}


% Create a new GUI Actor
GUI {
    start() -> spawn (fun() -> handleGUIRequest())
}

%  Create a new Backend Actor with initial value of 0 
Backend{
    start() -> spawn (fun() -> handleBackendRequest(0))
}


\end{lstlisting}

