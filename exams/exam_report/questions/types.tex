\section{Types}

\subsection{Problem 14: Rust type checking}

Consider the following program

\begin{lstlisting}
    fn f(write_ref: &mut Vec<i32>){
        write_ref[0] = 0;
    }

    fn g(read_ref : &Vec<i32>){
        println!("{}", read_ref[0])
    }
\end{lstlisting}

\subsubsection{Part A: Annotate owner}

Does the following program type check? Annotate for each line the current owner of the created vector. 

\begin{lstlisting}
    fn main(){
        let mut vec = vec![1,2,3];
        f(&mut vec);
        thread::spawn(move || g(&vec));
    }
\end{lstlisting}

\textbf{Solution}

The program does typecheck! The mutable reference is used before we move the closure. 
The owner is always \textit{vec} until the spawn of the new thread. Then we cannot use vec anymore due to the mov keyword. 

\begin{lstlisting}
    fn main(){
        let mut vec = vec![1,2,3]; // Owner: vec
        f(&mut vec); // Owner: vec
        thread::spawn(move || g(&vec)); // Owner: thread owns the vec now
    }
\end{lstlisting}


\subsubsection{Part B: Ownership typechecking}

\begin{lstlisting}
    // #1
    fn main(){
        let mut vec = vec![1,2,3];
        f(&mut vec);
        thread::spawn(move || g(&vec));
        g(&vec);
    }

    // #2
    fn main(){
        let mut vec = vec![1,2,3];
        f(&mut vec);
        g(&vec);
        g(&vec);
    }

    // #3
    fn main(){
        let mut vec = vec![1,2,3];
        f(&mut vec);
        g(&vec);
        g(&vec);
    }
\end{lstlisting}

\textbf{Solution}

For program \textit{1}, it does not typecheck. Because the ownership of the \textit{vec} to the thread. 
This means that after that line the vector cannot be used since it is moved out of scope. \\

For program \textit{2} and \textit{3} does typecheck. The core rule of rust borrowing is: 
\begin{quote}
    At any give time, you can either have:
    \begin{enumerate}
        \item One mutable reference 
        \item Any number of immutable references. 
    \end{enumerate}
\end{quote}

But, since we to a temporary borrow here, the ownership is returned at the end of the function call. It would be different if we created variables and not dropped them before calling new references. 
Then the borrow would stay active until the end of the program. But instead we return the borrowed memory after using it. 


\subsection{Problem 15: Linearity}

Consider the following Go-like code. Does it type-check? If no,
give the line of the statement where type-checking fails, the reason and the line where the
misused channel is declared. If yes, annotate for each declared channel (declared in the variables
c,d,e) the line where it is read and where it is written.

\begin{lstlisting}
    func main(){
        c = make(chan<!1,?1> int)
        d = make(chan<!1,?1> chan<!1,?1> int)
        e = make(chan<!1,?1> int);

        f = 0; 
        res = 0; 
        ret = 0; 

        go func{
            go func{
                d <-e; e <- 1; skip;
            }

            res = <-d;
            ret = 0; 
            if((<-res) < 0){
                ret = -1;
            }else{
                ret = 0; 
            }
            c <- ret*(<-e); skip;
        }

        f = <-c; skip;
    }
\end{lstlisting}

\textbf{Solution}

It fails on line 19. It reads E twice! We read the channel e into d which is read once (on line 12), and then we read e again on line 19. 
This will make the program not type checked. 

\begin{lstlisting}
    func main(){
        c = make(chan<!1,?1> int)
        d = make(chan<!1,?1> chan<!1,?1> int)
        e = make(chan<!1,?1> int);

        f = 0; // c -> !1,?1, d -> !1,?1, e -> !1,?1
        res = 0; // c -> !1,?1, d -> !1,?1, e -> !1,?1
        ret = 0; // c -> !1,?1, d -> !1,?1, e -> !1,?1

        go func{ // c -> !1,?1, d -> !1,?1, e -> !1,?1
            go func{ // c -> !1,?1, d -> !1,?1, e -> !1,?1
                d <-e; e <- 1; skip; // c -> !1,?1, d -> !0,?1, e -> !0,?1            }

            res = <-d; // c -> !1,?1, d -> !0,?0, e -> !0,?1
            ret = 0; // c -> !1,?1, d -> !0,?0, e -> !0,?1
            if((<-res) < 0){ // c -> !1,?1, d -> !0,?0, e -> !0,?0
                // Does not enter
                ret = -1;
            }else{ // c -> !1,?1, d -> !0,?0, e -> !0,?0
                ret = 0; // c -> !1,?1, d -> !0,?0, e -> !0,?0
            }
            // Error here! We are reading from channel e, but it has used up all read capabilities!
            c <- ret*(<-e); skip;
            
        }

        f = <-c; skip;
    }
\end{lstlisting}


\subsection{Problem 16: Binary Session types}

Consider the following Go-like code

\begin{lstlisting}
    func main(b bool, val1 int, val2 int){
        (c, c_dual) = make(chan T, chan T_)
        
        go f(c_dual);

        if (b){
            (r, r_dual) = make(chan S, chan S_)
            c <- l_1; 
            c <-r; 
            c <-val1; 
            r_dual <-val2;

            if (<-r_dual){
                println("success);
            }else{
                println("failure");

            }else{
                c <- abort; 
            }
        }
    }
\end{lstlisting}


\subsubsection{Part A: sessiontype T and S}

Give the session type for T and S so the program is well typed. \\

\textbf{Solution}

\[
T = \& \begin{cases}
    l_1: & !chan<S>.!int.0 \\
    abort: &  0
\end{cases}
\]

\[
S = ?int.!bool.0
\]

\subsubsection{Part B: Duals}

Give the duals of T and S . \\

\textbf{Solution}

\[
\hat{T} = \oplus \begin{cases}
    l_1: & ?chan<S>.?int.0 \\
    abort: &  0
\end{cases}
\]

\[
\hat{S} = !int.?bool.0\
\]

\subsubsection{Part C: subtype}

Give the subtype of T and $\hat{T}$ \\

\textbf{Solution}

We know the following: 

\[
T = \& \begin{cases}
    l_1: & !chan<S>.!int.0 \\
    abort: &  0
\end{cases}
\]


\[
\hat{T} = \oplus \begin{cases}
    l_1: & ?chan<S>.?int.0 \\
    abort: &  0
\end{cases}
\]

The subtypes are expressed as followed: 
\[T' <: T \]
\[\hat{T'} <: \hat{T} \]

The subtypes are expressed like this: 

\[
T' = \& \begin{cases}
    l_1: & !chan<S>.!int.0 \\
    abort: &  0 \\
    l: & 0
\end{cases}
\]


\[
\hat{T'} = \oplus \begin{cases}
    l_1: & ?chan<S>.?int.0 \\
\end{cases}
\]

