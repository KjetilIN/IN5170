\section{Exercise \#1}

The topic of the exercise is thinking concurrently and learning basic synchronization. 

\subsection{Problem 1: Parallelism and concurrency}

The notions of parallelism and concurrency, while
related, are not identical. Parallelism implies that executions “really” run at the same physical
time, whereas concurrent execution may happen on a mono-processor, where the fact that various
processes seem to happen simultaneously is just an “illusion” (typically an illusion maintained
by the operating system). Assume you have a mono-processor (and a single-core machine), so the CPU does not contain
parallel hardware. Under these circumstances, is it possible, that using concurrency makes
programs run faster? Give reason for your opinion. \\ 

\textbf{Solution:}

Concurrency can still make the program faster. 
It still makes all processes work faster since it allows scheduling of tasks to be optimal.
If a process would not run concurrently, it would execute instructions as seen.
This would in many cases be very slow.


\subsection{Problem 2: Synchronization}

Consider the following skeleton code:

\begin{lstlisting}
    string buffer;   # contains one line of the input
    bool done := false;
    process Finder { # find patterns
        string line1;
        while (true) {
            # wait for buffer to be full or done to be true;
            if (done) break;
            line1 := buffer;
            # signal that buffer is empty;
            # look for pattern in line1;
            if (pattern is in line1)
                write line1;
        }
    }
    process Reader { # read new lines
        string line2;
        while (true) {
            # read next line of input into line2 or set EOF after last line;
            if (EOF) {done := true; break;}
            # wait for buffer to be empty;
            buffer := line2;
            # signal that buffer is full;
        }
    }
\end{lstlisting}


\subsubsection{Part A}

Add missing code for synchronizing access to the buffer. Use \textit{await} statements for the synchronization. \\

\textbf{Solution}

When we write to the buffer, we need to make sure that the finder read the content of the buffer.
This is important so that we don't overwrite the content in the buffer before it is read. To solve this, we use
\textit{bufferEmpty} variable to signal when the buffer is empty. This will tell the \textit{Reader} to put something in the buffer.
We can use the same variable to signal that the buffer can be read. We do this with \textit{atomic} read statements such that we read and evaluate the statement in a single atomic action.  

\begin{lstlisting}
    string buffer;   # contains one line of the input
    bool done := false;
    bool bufferEmpty := false; 

    process Finder { # find patterns
        string line1;
        while (true) {
            <await bufferEmpty || done>
            if (done) break;
            line1 := buffer;

            # Signal empty buffer 
            bufferEmpty:= true;
         
            if (pattern is in line1)
                write line1;
        }
    }
    process Reader { # read new lines
        string line2;
        while (true) {
            if (EOF) {done := true; break;}

            <await bufferEmpty>
            buffer := line2;

            # signal that buffer is full;
            bufferEmpty := false; 
        }
    }
\end{lstlisting}

\subsubsection{Part B}

Extend your program so that it read two files and prints all the lines that contain pattern.
Identify the independent activities and use a separate process for each. 
Show all synchronization code that is required \\


\textbf{Solution}

The solution now means that two readers read into a single shared buffer. If there were more than one buffer, then we could have used our solution from part a.
But this task requires us to think about how to coordinate between the two reader. We introduce two \textit{done} variables to signal termination for both. 
The two readers read one file each, and then but the content in the same buffer. But there is an important point to remember here. 
In the original solution we check if the buffer is empty and then write to the buffer. This will not work here. For example, the two readers can both go past the atomic statement, which would allow then to both write the buffer.
To make this work, the checking as well as writing to the buffer must be atomic statements. This will ensure only one reader can add to the buffer.

\begin{lstlisting}
    string buffer;   # contains one line of the input
    bool done1 := false;
    bool done2 := false; 
    bool bufferEmpty := false; 

    process Finder { # find patterns
        string line1;
        while (true) {
            <await bufferEmpty || (done1 && done2)>
            if (done) break;
            line1 := buffer;

            # Signal empty buffer 
            bufferEmpty:= true;
         
            if (pattern is in line1)
                write line1;
        }
    }
    process Reader1 { # read new lines
        string line2;
        while (true) {
            if (EOF) {done1 := true; break;}

            <await bufferEmpty {
                buffer := line2;
                bufferEmpty := false; 
            }>
        }
    }

    process Reader2 { # read new lines
        string line3;
        while (true) {
            if (EOF) {done2 := true; break;}

            <await bufferEmpty {
                buffer := line3;
                bufferEmpty := false; 
            }>
        }
    }
\end{lstlisting}


\subsection{Problem 3: Producer-Consumer}

Consider the code of the simple producer-consumer problem below. Change it so that the variable p is local to the producer process
and c is local to the consumer process, not global. Hence, those variables cannot be used to
synchronize access to buf.


\begin{lstlisting}
    int buffer, p, c := 0;

    process Producer{
        int a[N];
        while(p < N){
            <await (p = c)>
            buffer := a[p];
            p:= p + 1;
        }
    }

    process Consumer{
        int b[N];
        while(c < N){
            <await (p>c) >
            b[c] := buffer;
            c:= c+1;
        }
    }
    
\end{lstlisting}

\textbf{Solution}

The task makes variable p and c local. This means that we cannot use them for synchronization. 
After making them local, we see the \textit{await} statements does not work anymore. 
To synchronize the access to the buffer we can use a boolean variable to control who should go next. 
The behavior from before was that the producer produces, but then waits for the consumer. It is alternating.
This means that our solution will also work here. 


\begin{lstlisting}
    int buffer; 

    # Variable to keep track of where the producer is 
    bool bufferEmpty := false; 

    process Producer{
        int p := 0;
        int a[N];
        while(p < N){
            <await (bufferEmpty = false)>
            buffer := a[p];
            p:= p + 1;
            bufferEmpty := true; 
        
        }
    }

    process Consumer{
        int c := 0; 
        int b[N];
        while(c < N){
            <await (bufferEmpty = true)>
            b[c] := buffer;
            c:= c+1;
            bufferEmpty := false; 
        }
    }
    
\end{lstlisting}


\subsection{Problem 4: Execution and atomicity}

Consider the following program:

\begin{lstlisting}
int x:= 0, y:= 0;

co 
    x:= x + 1  # S1
    x:= x + 2; # S2
||
    x:= x + 2; # P1
    y:= y - x; # P2
oc 
\end{lstlisting}


\subsubsection{Part A}

Suppose each assignment statement is implemented by a single machine instruction and
hence is atomic. How many possible executions are there? What are the possible final
values of x and y? \\

\textbf{Solution}

Since all statements are atomic we know there is a limited amount of execution.
But, \textit{S1} must execute before \textit{S2}, and \textit{P1} must execute before \textit{P2}.
The possible orders of executions then becomes: 

\begin{enumerate}
    \item S1, S2, P1, P2 $\to$ x:= 5, y:=-5  
    \item S1, P1, S2, P2 $\to$ x:= 5, y:=-5
    \item S1, P1, P2, S2 $\to$ x:= 5, y:=-3
    \item P1, S1, P2, S2 $\to$ x:= 5, y:=-3
    \item P1, S1, S2, P2 $\to$ x:= 5, y:=-5
    \item P1, P2, S1, S2 $\to$ x:= 5, y:=-2
\end{enumerate}

There is a total of 6 executions, and the end result is: \\
$(x:=5 \land  (y:=-5 \lor y:=-3 \lor y:=-2))$ 


\subsubsection{Part B}

Suppose each assignment statement is implemented by three atomic actions that load a
register, add or subtract a value from that register, then store the result. How many
possible executions are there now? What are the possible final values of x and y? \\


\textbf{Solution}

We know know there is three operations for each statement. We can simplify them as Read and Write. 
Read must happen before write. We can calculate executions by using this formula:

\begin{align*}
    E = \frac{(n*m)!}{m!^{n}}
\end{align*}

Where $n$ is the number of processes and $m$ are the number of atomic steps. In this case we have 2 processes and each process have 6 atomic steps: 

\begin{align*}
    E &= \frac{(n*m)!}{m!^{n}} \\
      &= \frac{(2*6)!}{6!^{2}} \\
      &= 924 \\
\end{align*}

There are a total of 924 possible executions. 

We know that X is never influenced by another processes other than S2. 
However, when the assignment is split into multiple instructions, we see that some instructions may be forgotten.
This may happen when a thread read variables and is then about to write, but before this, another thread has read and then adds. But this new value is not read. 
This means x could be $x \in {5,4,3,2}$. The y values would then be then be based on those values. But note that y could also -1

The end values are: 
\begin{align*}
    (x,y) \in (3,-3), (4,-4), (5, -5), (2, -2), (3, -1), (5, -2), (4, -2), (3, -2)
\end{align*}

For such a task: \textit{hard to find all solutions, and know when you have all solutions}. 



\subsection{Problem 5: Interleaving, non-determinism, and atomicity}

Consider the following program:

\begin{lstlisting}
    int x:= 2, y := 3;

    co
        <x := x + y> # S1
    ||
        <y := x * y> # S2
    oc
\end{lstlisting}


\subsubsection{Part A}

What are the possible values for X and Y? \\

\textbf{Solution}

Since both operation are atomic then we get the following executions: 
\begin{enumerate}
    \item S1, S2 $\to$ x:= 5, y:=15
    \item S2, S1 $\to$ x:= 8, y:=6
\end{enumerate}

Meaning the values are: \\
$(x:=5 \land y:=15) \lor (x:=8 \land y:=6)$


\subsubsection{Part B}

Suppose the angle brackets are removed and each assignment statement is now implemented by three atomic actions: 
read the variables, add or multiply, and write to available. What are the possible final values of x and y now? \\

\textbf{Solution} 

Since now we have to consider reading and writing operations, we get the following possible executions: 
\begin{enumerate}
    \item R1, W1, R2, W2 $\to$ x:= 5, y:= 15 
    \item R1, R2, W1, W2 $\to$ x:= 5, y:= 6 
    \item R1, R2, W2, W1 $\to$ x:= 5, y:= 6
    \item R2, R1, W1, W2 $\to$ x:= 5, y:= 6
    \item R2, R1, W2, W1 $\to$ x:= 5, y:= 6 
    \item R2, W2, R1, W1 $\to$ x:= 8, y:= 6 
\end{enumerate}

If both processes read before writing, then we get the same result, no matter what process reads first and what process writes first.
(Both read, then write). In the case of writing and then another process read, then we get different results. This leads to the following end result: \\
$(x := 5 \land (y:= 15 \lor y:= 6)) \lor (x:= 8 \land y:= 6)$


\subsection{Problem 6: AMO - At most once}

Consider the following program: 

\begin{lstlisting}
    int x:= 1, y:= 1;

    co 
        <x:= x + y> # S1
    ||
        y:= 0;      # S2
    ||
        x:= x - y   # S3
    oc
\end{lstlisting}


\subsubsection{Part A}

Do S1, S2 and S3 satisfy the requirements of the At-Most-Once Property? \\

\textbf{Solution} 

At most once property is used to check interleaving between the processes. 
\textit{Statements that fulfills the AMO-property can be considered atomic.} 


First \textit{S1}, it is atomic, and we therefor does not need to check the AMO property. 
For \textit{S2}, we see that it assigns a non-critical reference. This means that it satisfy the AMO property.
For \textit{S3}, it uses both x and y, which are both critical referees.
S1, S2 meets the AMO property, but S3 does not. 

\subsubsection{Part B}

What are the final values for x and y? Explain your answer. \\

\textbf{Solution} 

First, we see that are 3 processes which has either 1 or 2 atomic operations. 
Since S2 meets the AMO property, we can assume that it is atomic!

This will lead to the following executions:

\begin{enumerate}
    \item S1, S2, R3, W3 $to$ x:= 2, y:= 0
    \item S1, R3, S2, W3 $to$ x:= 2, y:= 1
    \item S1, R3, W3, S2 $to$ x:= 2, y:= 0
    \item S2, S1, R3, W3 $to$ x:= 1, y:= 0
    \item S2, R3, S1, W3 $to$ x:= 1, y:= 0
    \item S2, R3, W3, S1 $to$ x:= 1, y:= 0
    \item R3, W3, S1, S2 $to$ x:= 1, y:= 0
    \item R3, W3, S2, S1 $to$ x:= 0, y:= 0
\end{enumerate}


The result is then: \\
$ (y:=0 \land (x:=0 \lor x:=1 \lor x:=2)) \lor (x:=2 \land y:=1)$


\subsection{Problem 7: AMO and termination}

Consider the following program:

\begin{lstlisting}
    int x:=0, y:=10;

    co 
        while(x != y) x:=x+1;
    ||
        while(x != y) y:=y-1;
    oc
\end{lstlisting}

\subsubsection{Part A}

Do all parts of the program satisfy the AMO-property? \\

\textbf{Solution}

All assignments meets the AMO property in the program. 



\subsubsection{Part B}

Will the program terminate? Always? Sometimes? Never? \\

\textbf{Solution}

The program \textit{may} terminate (sometimes). It can happen when both processes see that x equal to y. 
S1 approaches from the bottom, and S2 approaches from the top. But there is a case where x is increased above current y value.
This will lead to both process incrementing/decrementing past each other. If this happens the program cannot terminate.  