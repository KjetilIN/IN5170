\section{Exercise \#2}

The topic of this exercise is synchronization of critical sections 

\subsection{Problem 1: Weak scheduling}

Consider the following program:

\begin{lstlisting}
    co
        <await (x >= 3) x:= x - 3> # P1
    ||
        <await (x >= 2) x:= x - 2> # P2
    ||
        <await (x = 1) x:= x + 5>  # P3
    oc
\end{lstlisting}


For which initial values of x does the program terminate (under weakly fair scheduling)?
What are the corresponding final values? Explain your answer. \\

\textbf{Solution}

\textit{Weak fairness ensures that if a process is enabled and remains enabled, it must eventually execute.} \\

We see that $x=1$ will terminate because we do P3, then P1 or P2. It will also terminate when $x=6$.
Both cases enables statements or makes the state stay enabled when first enabled. Any other values would not enable
P3. When $x=3 \lor x=4$ it may terminate based on the order of execution. 

To summarize

\begin{enumerate}
    \item $x <= 0$: Lower than 0, then all is blocking. No termination 
    \item $x = 1$: will start P3, and P2, P1 can execute at any order. Terminates 
    \item $x = 2$: will enter P2, and then x = 0. Then the other two will not terminate. No termination
    \item $x = 3$: May terminate. If enter P1 then it will not terminate (x will then be set be 0). If enter P2 it will terminate.
    \item $x = 4$; May terminate. 
    \item $x = 5$: No termination. X will be 0 after P1 and P2, and therefor will not execute P3
    \item $x = 6$: Always terminates. P1 and P2 will happen at any order and P3 will determinate. 
    \item $x >= 7$: Never terminate because x will always be greater than 1, so P3 will not happen. 
\end{enumerate}


\subsection{Problem 2: Weak scheduling}

Consider the following program:

\begin{lstlisting}
    co
        <await (x > 0) x:= x - 1> # P1
    ||
        <await (x < 0) x:= x + 2> # P2
    ||
        <await (x = 0) x:= x - 1>  # P
    oc
\end{lstlisting}


For which initial values of x does the program terminate (under weakly fair scheduling)?
What are the corresponding final values? Explain your answer. \\

\textbf{Solution}

