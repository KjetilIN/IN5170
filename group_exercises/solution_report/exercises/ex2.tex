\section{Exercise \#2}

The topic of this exercise is synchronization of critical sections 

\subsection{Problem 1: Weak scheduling}

Consider the following program:

\begin{lstlisting}
    co
        <await (x >= 3) x:= x - 3> # P1
    ||
        <await (x >= 2) x:= x - 2> # P2
    ||
        <await (x = 1) x:= x + 5>  # P3
    oc
\end{lstlisting}


For which initial values of x does the program terminate (under weakly fair scheduling)?
What are the corresponding final values? Explain your answer. \\

\textbf{Solution}

\textit{Weak fairness ensures that if a process is enabled and remains enabled, it must eventually execute.} \\

We see that $x=1$ will terminate because we do P3, then P1 or P2. It will also terminate when $x=6$.
Both cases enables statements or makes the state stay enabled when first enabled. Any other values would not enable
P3. When $x=3 \lor x=4$ it may terminate based on the order of execution. 

To summarize

\begin{enumerate}
    \item $x <= 0$: Lower than 0, then all is blocking. No termination 
    \item $x = 1$: will start P3, and P2, P1 can execute at any order. Terminates 
    \item $x = 2$: will enter P2, and then x = 0. Then the other two will not terminate. No termination
    \item $x = 3$: May terminate. If enter P1 then it will not terminate (x will then be set be 0). If enter P2 it will terminate.
    \item $x = 4$; May terminate. 
    \item $x = 5$: No termination. X will be 0 after P1 and P2, and therefor will not execute P3
    \item $x = 6$: Always terminates. P1 and P2 will happen at any order and P3 will determinate. 
    \item $x >= 7$: Never terminate because x will always be greater than 1, so P3 will not happen. 
\end{enumerate}


\subsection{Problem 2: Weak scheduling}

Consider the following program:

\begin{lstlisting}
    co
        <await (x > 0) x:= x - 1> # P1
    ||
        <await (x < 0) x:= x + 2> # P2
    ||
        <await (x = 0) x:= x - 1>  # P3
    oc
\end{lstlisting}


For which initial values of x does the program terminate (under weakly fair scheduling)?
What are the corresponding final values? Explain your answer. \\

\textbf{Solution}

\begin{enumerate}
    \item $x<-2$ does not terminate because after P2, then there is nothing that enables P1 and P3
    \item $x=-2$ does not terminate because after P2, P3, then P1 will not be enabled.
    \item $x=-1$ terminates under P2, P1, P3
    \item $x=0$ terminates under P3, P2, P1
    \item $x=1$ terminates under P1, P3, P2
    \item $x=2$ does not terminate, because after P1, x is 1 and there is no condition that gets enabled
    \item $x>2$ does not terminate for the same reasons as x equal to 2.
\end{enumerate}

The final values are: $x\in [-1,1]$



\subsection{Problem 3: Termination under scheduling strategy}

Consider the following program:

\begin{lstlisting}
    int x := 10;
    bool x:= true;

    co 
        <await (x=0);> c:=false # P1
    ||
        while(c) <x:=x-1>       # P2
    oc
\end{lstlisting}


\subsubsection{Part A: termination under weak fairness}

Does the program terminate under weak scheduling? \\

\textbf{Solution}

Weak scheduling ensures that if a condition is enabled and stays enabled, then it will terminate. 
In this case we see P2 is enables and starts to decrement x. P1 is only enabled when $x=0$. This may happen, but we see that it get enabled until P2 executes again such that x is less than 0.
This means that it may terminate, but it is not guaranteed.

\subsubsection{Part B: termination under strong fairness}

Does the program terminate under weak fairness? \\

\textbf{Solution}

Strong fairness ensures termination if it enables infinitely often. This is not the case. After x being decremented below 0, it may never become 0 again. 
This means that the program may terminate but there is no guarantee. 


\subsubsection{Part C}

Program is not extended to:

\begin{lstlisting}
    int x := 10;
    bool x:= true;

    co 
        <await (x=0);> c:=false      # P1
    ||
        while(c) <x:=x-1>            # P2
    ||
        while(c) {if (x<0) <x:=10>;} # P3
    oc
\end{lstlisting}

Does it now terminate under weak or strong fairness? \\ 


\textbf{Solution}

Under \textit{weak fairness} it may terminate. Again for the same reasons as before. But again weak fairness only ensures termination when the condition is enabled and stays enabled.
In this case it will be enabled, but not stay enabled. 

However, under \textit{strong fairness} it will terminate. This is because the new branch will ensure that the condition is infinitely enabled. 
Thus it terminates. 
