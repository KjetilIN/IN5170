\section{Exercise \#6}

Topic is types for concurrency


\subsection{Problem 1: Extend type system}

Extend the following syntax for statements s such that the syntax also defines
return statements, which enable to return the value defined by an expression. Given the syntax
for types T, define a typing rule for the return statements. You can assume that the typing rules
given from the lecture already exist

\begin{equation}
    \begin{aligned}
    & \mathrm{s}::=\mathrm{v}=\mathrm{e} ; \mathrm{s}|\mathrm{~T} \mathrm{v}=\mathrm{e} ; \mathbf{s}| \text { skip } \mid \text { if }(\mathrm{e})\{\mathbf{s}\} \mathrm{s} \\
    & \mathrm{e}::=n \mid \text { true } \mid \text { false }|\mathrm{e}+\mathrm{e}| \mathrm{e} \wedge \mathrm{e}|\mathrm{e} \leq \mathrm{e}| \mathrm{v} \\
    & \mathrm{~T}::=\text { Int } \mid \text{Bool} \mid \text{Unit}
    \end{aligned}
\end{equation}

\textbf{Solution}

To give this the correct syntax, we extend the statements \textit{s} to be: 

\begin{align*}
    \mathrm{s}::=\mathrm{v}=\mathrm{e} ; \mathrm{s}|\mathrm{~T} \mathrm{v}=\mathrm{e} ; \mathbf{s}| \text { skip } \mid \text { if }(\mathrm{e})\{\mathbf{s}\} \mathrm{s} \mid \text{return e} \\
\end{align*}

Because \textit{return} is a statement, and we return an expression. 


\subsection{Problem 2: Function Call Type System}

We extend the syntax given in Problem 1 such that we can perform function calls, 
where a function $f$ maps an argument of type $T_1$ to type $T_2$, i.e., $f: \mathrm{T}_1 \mapsto \mathrm{~T}_2$. 
An example of a function would be a successor function succ: Int $\mapsto$ Int that increases a given integer by one. 
Similar to variables, functions are also stored in the typing environment $\Gamma$. 
An example of $\Gamma$ with a function would be $\Gamma=\{$ succ $\mapsto$ Int $\mapsto$ Int $\}$ 
We update the syntax as follows:

\[
\mathrm{e}::=n \mid \text { true } \mid \text { false }|\mathrm{e}+\mathrm{e}| \mathrm{e} \wedge \mathrm{e}|\mathrm{e} \leq \mathrm{e}| \mathrm{v} \mid \mathrm{f}(\mathrm{e})
\]

The type system of Example 1 is extended by the following typing rule:

\[
\frac{\Gamma(\mathrm{f})=\mathrm{T}_1 \mapsto \mathrm{~T}_2 \quad \Gamma \vdash \mathrm{e}: \mathrm{T}_1^{\prime} \quad \mathrm{T}_1^{\prime}<: \mathrm{T}_1}{\Gamma \vdash \mathrm{f}(\mathrm{e}): \text { Unit }} \text { function-call }
\]

Does this extension of the type still ensure type soundness considering also the typing rules in the lecture? 
If not, can you give an example that shows the violation of type soundness. \\


\textbf{Solution}

This violates types soundness because a function-call is an expression and therefor not a statement. In the original typing rule it is typed as an statement.
To change this, we use $T_2$ as the type of the function call, where $T_2$ is the return type. 
We can also validate this with 


\subsection{Problem 3: Annotate Go Program for different type systems}

Consider the following Go code: 

\begin{lstlisting}[language=go]
    func add(inp chan int, fin chan bool, res chan int){
        sum := 0
        for{
            select{
                case num:= <-inp:{
                    sum = sum + num; 
                }

                case <-fin:{
                    res <- sum
                }
            }
        }
    }

    func provide(inp chan int, fin chan bool){
        inp <- 9
        inp <- 3
        fin <- true
    }


    func main(){
        input := make(chan int)
        finish := make(chan bool)
        res := make(chan int)

        go add(input, finish, res)
        go provide(input, finish)
        <- res
    }
\end{lstlisting}


Annotate the channel types such that they support the following: 

\subsubsection{Modes}

Modes describe what channels are allowed to do. A channel can either send, receive or both. 
The send operations is denoted with \textit{!} and receive operations are denoted with \textit{?}.
For the program we can annotate it the following way: 

\begin{lstlisting}[language=go]
    func add(inp chan? int, fin chan? bool, res chan! int){...}

    func provide(inp chan! int, fin chan! bool){...}

    func main(){
        input := make(chan?! int)
        finish := make(chan?! bool)
        res := make(chan?! int)
        ....
    }
\end{lstlisting}

\subsubsection{Linear Types}

Linear usage types keep track of how many times channel can do operations. 
With linear types we specify how many times an operation is allowed of each mode. 
\textit{w} denotes more than two times and are noted as arbitrary amount of time. 
We denote this with the following: 

\begin{lstlisting}[language=go]
    func add(inp chan<?.w> int, fin chan<?.1> bool, res chan<!.1> int){...}

    func provide(inp chan<!.w> int, fin chan<!.1> bool){...}

    func main(){
        input := make(chan<?.w.!.w> int)
        finish := make(chan<?.1.!.1> bool)
        res := make(chan<?.1.!.1> int)
        ....
    }
\end{lstlisting}

\subsubsection{Usage Types}
Usage types are denoted with how many times and the order of operations (the order of operation is very powerful thing to denote). 
The orders of operations are noted, and at the end there is 0 to make the channel not usable.
We use the \textit{+} operator when we create a channel to denote that the different types used in a channel. 
It shows the types of the channel when it is used by more than one thread. 
Unlike linear types, we do not use the \textit{w} notation for arbitrary amount of operations. 

\begin{lstlisting}[language=go]
    func add(inp chan<?.?.0> int, fin chan<?.0> bool, res chan<!.0> int){...}

    func provide(inp chan<!.!.0> int, fin chan<!.0> bool){...}

    func main(){
        input := make(chan<?.?.0 + !.!.0> int)
        finish := make(chan<?.0 + !.0> bool)
        res := make(chan<?.0 + !.0> int)
        ....
    }
\end{lstlisting}
