\section{Exercise \#7}

Topic of this exercise is creating a simple typing system. The whole exercise has all problems connected as one. 
The following is already defined: 

In this exercise we are going to design a simple uniqueness type system that ensures that for every pair of integers there is only one variable that can be used to read its value. A pair of integers can be created with $v=\operatorname{Pair}(x, y)$, its first value can be updated with $v . f s t=a$ and its second value with $v . s n d=b$ for integers $x, y, a, b$. Let $V$ be the set of all variables. Our language has statements and expressions defined by the following grammar, with $f \in\{f s t, s n d\}, v \in V, n \in N$.

\[
\begin{aligned}
& s::=T v=\operatorname{Pair}(e, e) ; s|T v=e ; s| v=e ; s \mid v \cdot f=e ; s \\
& \mid \text { if(e)\{s\} else }\{s\} s \mid \text { go\{s\} s|skip } \\
& e::=n|v| v \cdot f \mid e \geq 0
\end{aligned}
\]


\subsection{Problem 1: Type syntax}


Give a type syntax for our system. We need integers, booleans, and the unit type.
Additionally, we need a type for pairs that encodes whether the variable with this type can be
used for access or not. \\

\textbf{Solution}

We can define the types with the following syntax:

\[
 T ::= \text{Int} \mid \text{Bool} \mid \text{Unit} \mid \text{Pair}_n, n \in [0,1] 
\]

We define all the types given from the text. To make the pair accessible or not, we use $n$ which is either 0 or 1. 
0 would mean that it is not used for access, and 1 for being used for access. 

\subsection{Problem 2: Defining environments}

Define the signature of the type environment and define a split into two smaller
environments on it, i.e., an operation such that access to a Pair is only allowed within one of
the smaller environments.\\

\textbf{Solution}

Splitting the environment would be done like this: 
\[
\Gamma = \Gamma_1 + \Gamma_2
\]

Where the splitting of the environment would lead to only access in one of the environments. We define it like this:

if $\Gamma(x)=$ Pair $_n$ then $\Gamma_1(x)=\operatorname{Pair}_a \wedge \Gamma_2(x)=$ Pair $_b \wedge a+b=n$ else $\Gamma_1(x)=\Gamma_2(x)=\Gamma(x)$

\subsection{Problem 3: Unrestricted environment}

Does the system require a notion of unrestricted environments? If yes, define it, if
no, argue why. \\

\textbf{Solution}

We typically use unrestricted for environments such as channels to show that all capabilities of the channels in the environment is used!
However, since there are no request for unrestricted the environment we don't need to do so. 
We could have made the program use all environments, but most modern day compilers give a warning when this is the case, and not a error (for example C compliers or Rust).


\subsection{Problem 4: Type system for expressions}

Give a type system for expressions that ensures that only pairs that we have the
sole reference to can be accessed. Note: There are no nested Pairs. You can have different rules depending on whether you
operate on a Pair variable or not.

\textbf{Solution}

The key here is the last rule. It is a rule for when we access the pair that the variable is accessible. 

\begin{equation}
    \begin{gathered}
    \overline{\Gamma \vdash n: \text { int }} \\ \\
    \frac{\Gamma(v)=T}{\Gamma \vdash v: T} \\ \\
    \frac{\Gamma \vdash e: \text { int }}{\Gamma \vdash e \geq 0: \text { bool }} \\ \\
    \frac{\Gamma(v)=\text { Pair }_1}{\Gamma \vdash v \cdot f: \text { int }}
    \end{gathered}
\end{equation}

\subsection{Problem 5: Type system for statements}

Give a type system for statements that ensures that only pairs that we have the
sole reference to can be accessed. At which statements do you need to split the environment?
Where do you need to update the environment?
Note: There are no nested Pairs. You can have different rules depending on whether you operate on a Pair variable or not. \\

\textbf{Solution}

\[
\begin{aligned}
    & \frac{\Gamma\left[v \mapsto \text { Pair }_1\right] \vdash s: \text { Unit } \quad \Gamma \vdash e_1: \text { int } \quad \Gamma \vdash e_2: \text { int }}{\Gamma \vdash \operatorname{Pair}_1 v=\operatorname{Pair}\left(e_1, e_2\right) ; s: \text { Unit }} \\
    & \frac{\Gamma\left[v \mapsto \text { Pair }_1\right]\left[w \mapsto \text { Pair }_0\right] \vdash s: \text { Unit }^{\Gamma} \quad \Gamma(w)=\text { Pair }_1}{\Gamma \vdash \text { Pair }_1 v=w ; s: \text { Unit }} \\
    & \frac{\Gamma[v \mapsto T] \vdash s: \text { Unit } \quad \Gamma \vdash e: T \quad T \neq \text { Pair }_n}{\Gamma \vdash T v=e ; s: \text { Unit }} \\
    & \frac{\Gamma\left[w \mapsto \text { Pair }_0\right]\left[v \mapsto \text { Pair }_1\right] \vdash s: \text { Unit }^{\Gamma} \quad \Gamma(w)=\text { Pair }_1 \quad \Gamma(v)=\text { Pair }_n}{\Gamma \vdash v=w ; s: \text { Unit }} \\
    & \frac{\Gamma \vdash s: \text { Unit } \quad \Gamma \vdash e: T \quad \Gamma(v)=T \neq \text { Pair }_n}{\Gamma \vdash v=e ; s: \text { Unit }} \\
    & \frac{\Gamma(v)=\text { Pair }_1 \quad \Gamma \vdash e: \text { int } \quad \Gamma \vdash s: \text { Unit }}{\Gamma \vdash v . f=e ; s: \text { Unit }} \\
    & \frac{\Gamma \vdash e: \text { bool } \Gamma \vdash s_1 ; s_3: \text { Unit } \quad \Gamma \vdash s_2 ; s_3: \text { Unit }}{\Gamma \vdash \text { if }(e)\left\{s_1\right\} \text { else }\left\{s_2\right\} s_3: \text { Unit }} \\
    & \frac{\Gamma=\Gamma_1+\Gamma_2 \quad \Gamma_1 \vdash s_1: \text { Unit } \quad \Gamma_2 \vdash s_2: \text { Unit }}{\Gamma \vdash \text { go }\left\{s_1\right\} s_2: \text { Unit }} \\
    & \overline{\Gamma \vdash \text { skip : Unit }}
\end{aligned}
\]


The split is only at the go statement starting a new thread. 
There is no split at $if$, as we allow it to be accessed multiple times. 
The order is realized at the copy assignment for pairs. 
The old variable loses its capability to be accessed, and it is transferred to the new one. 
Similarly, the if statement does not type its branches in isolation, but with the continuation.

\subsection{Problem 6: Type check the following program}


\textit{Not Done}